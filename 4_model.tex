\section{Modelling of Handover Manipulations}

- present a short version of the deceleration phase (previous paper)
- explain why you decide to resort to this new approach

\subsection{Deceleration Phase}

Let $x \in D \subset \mathbb{R}^{+}$ denote the distance of the human wrist towards the handover meeting point. Consider a behavior encoded as a state-dependent dynamical system (DS)
%
\begin{equation}
\dot{x} = \pmb{\textnormal{f}}(x)
\label{eq:1}
\end{equation}
where $\pmb{\textnormal{f}}:\mathbb{R}^{+} \to \mathbb{R}^{+}$ is a continuous and continuously differentiable function, with a single equilibrium point ${\dot{x}_d}^* = \pmb{\textnormal{f}}(x^*)$. $x^{*}$ is set at the origin and it is globally asymptotic stable such that $\dot{x}^* = \textnormal{f}(x^{*}) = 0$ which is guaranteed under a Lyapunov function $V(x):\mathbb{R}^{+} \rightarrow \mathbb{R}^{+}$.

Our approach defines each ``carefulness'' condition, \textit{careful} and \textit{not careful}, as two distinct DS. Each DS is encoded using Gaussian Mixture Models (GMM) which defines a joint distribution function $\mathcal{P}({{x}^{t}}_n, {\dot{x}^{t}}_n | \Theta) = \sum_{k=1}^{K} \pi^{k} \mathcal{N}({{x}^{t}}_n, {\dot{x}^{t}}_n, \mu^{k}, \Sigma^{k})$ over the data as mixture of $K$ Gaussian distributions \cite{khansari2011learning}, where $\pi^{k}$, $\mu^{k}$, and $\Sigma^{k}$ are, respectively, the prior component, mean, and covariance matrix of the $k$th Gaussian. $x_n^t$ is $n$th trajectory of $x$ at time $t$, and $\dot{x}_n^t$ is its derivative. Fig. \ref{fig:motion} illustrates the position ($x$) and velocity ($\dot{x}$) relations for \textit{careful} and \textit{not careful} motions. To compute the DS from Eq. (\ref{eq:1}) the posterior mean of $\mathcal{P}({\dot{x}^{t}}_n|{{x}^{t}}_n)$ is estimated which approximates it to:
%
\begin{equation}
\hat{\dot{x}} = \sum_{n=1}^{K} h^{k}(x) (\Sigma^{k}_{\dot{x}x}(\Sigma^{k}_{xx})^{-1} (x - \mu^{k}_{x}) + \mu^{k}_{\dot{x}})
\label{eq:2}
\end{equation}
where $h^{k}(x) = \frac{\pi^{k} \mathcal{N}({{x}^{t}}, {\dot{x}^{t}}, \mu^{k}, \Sigma^{k})}{\sum_{i=1}^{K} \pi^{k} \mathcal{N}({{x}^{t}}_n, {\dot{x}^{t}}_n, \mu^{i}, \Sigma^{i})}$, $h^{k}(x) > $ 0, and $\sum_{n=1}^{K} h^{k}(x)$ = 1. The GMMs are computed using the stable estimator of dynamical systems (SEDS) approach \cite{khansari2011learning}. 

\subsection{Acceleration Phase}

- present the new approach of the acceleration phase

- show the results of the table of deceleration vs acceleration

\subsection{Classification}

\subsection{Comparing the two approaches}

\begin{table} 
\centering 
\resizebox{\columnwidth}{!}{%
\begin{tabular}{l l c c c c c c} 
\toprule % Top horizontal line
 & & & \multicolumn{5}{c}{\textbf{Carefulness Detection}} \\ 
\cmidrule(l){4-7} 
\textbf{Type of Cup} &  &  & \multicolumn{2}{c}{Acceleration Phase} & \multicolumn{2}{c}{Deceleration Phase} &\\ % Column names row
\cmidrule(l){4-7} 
\textbf{Train} & \textbf{Test} & \diagbox{Predicted}{Real} & Empty & Full & Empty & Full &\\ % Column names row
\midrule % In-table horizonta0l line
\multirow{2}{*}{Red Cup}  & \multirow{2}{*}{Red Cup} & Not Careful & 0.77 & \textcolor{Grey}{0.17} & \textbf{0.82} & \textcolor{Grey}{0.42} \\
  &  & Careful & \textcolor{Grey}{0.23} & \textbf{0.83} & \textcolor{Grey}{0.18} & 0.58 \\ 
\cmidrule(l){2-7} 
\multirow{2}{*}{Champagne} & \multirow{2}{*}{Champagne} & & 0.4 & \textcolor{Grey}{0} & \textbf{1} & \textcolor{Grey}{0.5} \\ 
  &  &  & \textcolor{Grey}{0.6} & \textbf{1} & \textcolor{Grey}{0} & 0.5 \\ 
\cmidrule(l){2-7} 
\multirow{2}{*}{Transparent Cup} & \multirow{2}{*}{Transparent Cup}  & & \textbf{0.8} & \textcolor{Grey}{0.33} & 0.71 & \textcolor{Grey}{0.26}\\ 
&  &  & \textcolor{Grey}{0.2} & 0.67 & \textcolor{Grey}{0.29} & \textbf{0.74} \\ 
\cmidrule(l){2-7} 
\multirow{2}{*}{Red Mug} & \multirow{2}{*}{Red Mug}  & & \textbf{0.5} & \textcolor{Grey}{0} & 0 & \textcolor{Grey}{0.25} \\
 &  &  & \textcolor{Grey}{0.5} & \textbf{1} & \textcolor{Grey}{1} & 0.75 \\ 
\cmidrule(l){2-7} 
\multirow{2}{*}{Wine Glass} & \multirow{2}{*}{Wine Glass}  & & 0.57 & \textcolor{Grey}{0.4} & \textbf{0.6} & \textcolor{Grey}{0}\\ 
&  &  & \textcolor{Grey}{0.23} & 0.6 & \textcolor{Grey}{0.4} & \textbf{1} \\ 

\bottomrule % Bottom horizontal line
\end{tabular}
}
\label{tab:accele_vs_decele}
\caption{Training set: One cup type; Testing set: Same cup type.}
\end{table}

\subsecion{Discussion}


