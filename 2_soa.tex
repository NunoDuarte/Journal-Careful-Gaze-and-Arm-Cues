\section{Relevant Work}

what papers to mention:

- here in the soa you focus on the work related, the neuroscience behind it, the psychology behind it, the non-verbal communication behind it, the different manipulation works, the different handover approaches, the affordances of objects, the impact that weight and object properties has on the human manipulation, mention your previous work

 \cite{sciutti_understanding_2014}
By action observation we can infer the goal of the action or even the object's weight. This implicit understanding is developed early in childhood and is supposedly based on a common motor repertoire between the cooperators. Results: subjects can reach a performance in weight recognition from robot observation comparable to that obtained during human observations with no need of training. 3 experiments are made in this paper - neural (doing nothing), human action/movie watching - active (while object lifting), robot action - neutral. Known (know the weight and grasp before), Human condition (unknown weight), Interaction (not grasping before). 
Discussion: robot lifting with standard motion => humans can not estimate weight => obviously! Robot proportional allows! -> weight reading derives a generalization of the skills normally adopted in HHI. Human explicit kinematic whether it is a human or a robot. -> However the robot must exhibit familiar motor behaviours to the human. Otherwise it could cause detrimental effects => uncanny valley. How strong is the human influences how we estimate the object. 
 
 \cite{senot_effect_2011}
Explicit weight-related information => written labels on the objects. - and if there is a conflict between label and object weight. 
Discussion: Hidden (object hidden-only kinematic cues) - significant modulation of MEPs amplitude according with the force required to hold the object. Visible (visible object and kinematic cues) - not significant different. Labels (labels on the object) - influence the observers - same label for light and heavy object then MEPs amplitude was not significantly different. - different labels => same as in Hidden or Visible conditions. 
Previous results: high-level semantic cues, such as labels, may influence low-level motor behaviour during execution. - Any time a conflict is present between explicit and implicit movement-related information, the observers motor system stops its mirroring of the observed action -> mirror mechanisms is not blind to semantic info. - predominance of kinematic cues over intrinsic object properties in coding force required to execute an observed lifting action. 

 \cite{sciutti_development_2019}
Infer the weight of unknown object - intrinsic feature not directly accessible through vision. Judge weight of objects after watching videos of an actor lifting. At age 6 they can estimate difference in weight by observing different videos of objects. At age 10 the variability of estimation decreases and the accuracy increases. However they usually underestimate the weight. Adults are better at estimating. Improvement of performance reflects the development of children's motor control. - Children's stature was more reliable than their chronological age -> Body heights more directly related to growth than age. - Features important to estimate weight? Temporal and spatial-temporal features of the movements. => duration and absolute velocity. 

 \cite{lastrico_careful_2021}
I NEED TO READ THIS PAPER

previous work 
