\section{Introduction}

what you talk about:

- in the intro you introduce the idea, the challenge, the different ways that people have tried to use to solve it, and your proposal
- you mention the CV people who are trying to detect liquid levels or if the cup has liquid or not from images or videos. There is a problem with occlusions, different types of cups, color of cups, opaque cups (where it is impossible to tell). I propose another way that helps solve that problem

- work on CV
\cite{yu_fill_2015} infer the cup level of water
\cite{mottaghi_see_2017} reasoning from the level of liquid in containers

- learning from demonstration
\cite{santina_learning_2019} LfD

\cite{mayer_walking_2012} walking with coffee mugs in our daily lives. Whether it is a feedback (closed-loop) control system, i.e. when human being "identifies" the resonant sloshing frequency and then performs a targeted suppression of the resonant mode, or, an open-loop control system, i.e. when a human being simply becomes more careful about carrying a cup regardless of natural frequency of the fluid in it. -> may depend on the individual.

we argue that, although there is some degree of personal choice for each human, the majority of people try to solve this problem using the latter choice. In other words, to simplify the problem of moving from point A to point B with a mug full of coffee we instead just adjust our motion, either by slowing down or reducing the jerkiness of the movements, in order to prevent spilling. And this constitutes, for us, as someone being careful when manipulating an object. 


